\documentclass[a4paper]{article}
\usepackage[spanish]{babel}
\usepackage[utf8]{inputenc}
\usepackage{fancyhdr}
\usepackage{charter}   % tipografía
\usepackage{graphicx}
\usepackage{makeidx}

\usepackage{float}
\usepackage{amsmath, amsthm, amssymb}
\usepackage{amsfonts}
\usepackage{sectsty}
\usepackage{wrapfig}
\usepackage{listings} % necesario para el resaltado de sintaxis
\usepackage{caption}
\usepackage{placeins}

\usepackage{hyperref} % agrega hipervínculos en cada entrada del índice
\hypersetup{          % (en el pdf)
    colorlinks=true,
    linktoc=all,
    citecolor=black,
    filecolor=black,
    linkcolor=black,
    urlcolor=black
}

\input{codesnippet}
\input{page.layout}
\usepackage{underscore}
\usepackage{caratula}
\usepackage{url}
\usepackage{color}
\usepackage{clrscode3e} % necesario para el pseudocodigo (estilo Cormen)




\begin{document}

\lstset{
  language=C++,                    % (cambiar al lenguaje correspondiente)
  backgroundcolor=\color{white},   % choose the background color
  basicstyle=\footnotesize,        % size of fonts used for the code
  breaklines=true,                 % automatic line breaking only at whitespace
  captionpos=b,                    % sets the caption-position to bottom
  commentstyle=\color{red},    % comment style
  escapeinside={\%*}{*)},          % if you want to add LaTeX within your code
  keywordstyle=\color{blue},       % keyword style
  stringstyle=\color{blue},     % string literal style
}

\thispagestyle{empty}
\materia{Sistemas Operativos}
\submateria{Segundo Cuatrimestre de 2015}
\titulo{Título}
\subtitulo{Subtítulo}
\integrante{Confalonieri, Gisela Belén}{511/11}{gise_5291@yahoo.com.ar} % por cada integrante (apellido, nombre) (n° libreta) (e-mail)
\integrante{Mignanelli, Alejandro Rubén}{609/11}{minga_titere@hotmail.com} 
\integrante{Sabarros, Ian}{661/11}{iansden@live.com}

\maketitle
\newpage

\thispagestyle{empty}
\vfill
%\begin{abstract}
%    \vspace{0.5cm}
%	
%
%\end{abstract}

\thispagestyle{empty}
\vspace{1.5cm}
\tableofcontents
\newpage

%\normalsize
 
\newpage

\section{Ejercicio 1}

\vspace*{0.3cm}
\section{Ejercicio 1 - TaskConsola}

Se programó un tipo de tarea llamado {\tt TaskConsola}, que se ocupa de realizar {\it n} llamadas bloqueantes, cada una con una duración al azar comprendida entre los valores {\it bmin} y {\it bmax}.

\subsection{Algoritmo}

La Figura \ref{cod-tconsola} muestra el pseudocódigo de esta tarea.

\begin{figure}[!htb]
\begin{codebox}
\Procname{$\proc{TaskConsola}(n,bmin,bmax)$}
\li \For $i \leftarrow 0 .. n$
\li \Do 	$ciclos\_bloqueo \leftarrow$ valor ``random'' en $[bmin,bmax]$
\li 		bloquear durante $ciclos\_bloqueo$ ciclos
\End
\end{codebox}
\caption{Pseudocódigo TaskConsola}\label{cod-tconsola}
\end{figure}

Para el cálculo del valor ``random'' se utilizó la función {\tt rand()} provista por la librería {\tt stdlib} de C++.

\subsection{Pruebas}

Se creó un lote con 3 tareas de tipo {\tt TaskConsola} para correr en el simulador con un scheduler FCFS.  La Figura \ref{fig-1} muestra el resultado de dicha simulación.

\begin{figure}[!htb]
\begin{center}
  \includegraphics[scale=0.45]{imagenes/ej1.png}
\end{center}
\caption{Simulación lote TaskConsola con FCFS, 1 núcleo, 1 ciclos de cs}\label{fig-1}
\end{figure}
\vspace*{0.6cm}

\section{Ejercicio 2}

\vspace*{0.3cm}
\section{Ejercicio 2 - El lote de Rolando}

Se escribió un lote de tareas para simular la siguiente situación:

\begin{itemize}
	\item Correr un algoritmo que hace un uso intensivo de la CPU por 100 ciclos sin realizar llamadas bloqueantes.
	\item Escuchar una canción, que realiza 20 llamadas bloqueantes con una duración variable entre 2 y 4 ciclos.
	\item Navegar por internet, que realiza 25 llamadas bloqueantes con una duración variable entre 2 y 4 ciclos.
\end{itemize}

Estas tareas se ponen a correr en el instante 0, 1 y 2 respectivamente, y Rolando espera que ejecute11n en simultáneo.

\subsection{Pruebas}

Se ejecutó el simulador utilizando el algoritmo FCFS para 1 y 2 núcleos, con un cambio de contexto de 4 ciclos.  Las Figuras \ref{fig-rol1core} y \ref{fig-rol2core} muestran el resultado de ambas simulaciones.

\begin{figure}[!htb]
\begin{center}
  \includegraphics[scale=0.45]{imagenes/ej2-1core.png}
\end{center}
\caption{Simulación lote de Rolando con FCFS, 1 núcleo, 4 ciclos de cs}\label{fig-rol1core}
\end{figure}

\begin{figure}[!htb]
\begin{center}
  \includegraphics[scale=0.45]{imagenes/ej2-2core.png}
\end{center}
\caption{Simulación lote de Rolando con FCFS, 2 núcleos, 4 ciclos de cs}\label{fig-rol2core}
\end{figure}

En el caso de la Figura \ref{fig-rol1core}, la latencia de cada proceso es:

\begin{itemize}
	\item {\bf Tarea 0:} latencia 4
	\item {\bf Tarea 1:} latencia 108
	\item {\bf Tarea 2:} latencia 202
\end{itemize}

Estos valores nos dan un promedio aproximado de 30,3.
\vspace*{0.5cm}

Por otro lado, en el caso de la Figura \ref{fig-rol2core}, la latencia de cada proceso es:

\begin{itemize}
	\item {\bf Tarea 0:} latencia 4
	\item {\bf Tarea 1:} latencia 4
	\item {\bf Tarea 2:} latencia 83
\end{itemize}

Estos valores nos dan un promedio aproximado de 100,6.
\vspace*{0.5cm}

En el caso en el que Rolando se viera obligado a utilizar una computadora con un solo núcleo, como se utiliza un Sheduler del tipo FCFServe, no podría escuchar su canción preferida MIENTRAS corre el algoritmo, ya que las tres tareas se ejecutarán secuencialmente.  En el caso de poder utilizar una computadora con dos núcleos, podría correrse el algoritmo en uno de ellos y la canción en el otro, y Rolando podría trabajar a gusto.
\vspace*{0.6cm}

\section{Ejercicio 3}

\vspace*{0.3cm}
\section{Ejercicio 3 - TaskBatch}

Se programó un tipo de tarea llamado {\tt TaskBatch}.  Este tipo de tarea realiza {\it cant_bloqueos} llamadas bloqueantes en momentos elegidos pseudoaleatoriamente, y cada bloqueo dura 1 ciclo.  Además, utiliza el el CPU durante {\it total_cpu} ciclos, incluyendo el tiempo necesario para lanzar las llamadas bloqueantes, pero no el tiempo en el que el proceso permanece bloqueado.

\subsection{Algoritmo}

La idea de nuestro algoritmo se basa en decidir, a cada ciclo y de manera pseudoaleatoria, si se realiza un bloqueo o no.  Para tomar esta decisión vamos a tomar un valor entero ``random'' entre 0 y 1 (1 para bloquear y 0 para no hacerlo), nuevamente utilizando la función {\tt rand()} de C++.

Como cada llamada bloqueante consume 1 ciclo de utilización de CPU, podemos decir que {\it total_cpu} incluye {\it cant_bloqueos} ciclos destinados a las llamadas bloqueantes, y el resto son usos ``puros'' de CPU.  Por este motivo, la decisión pseudoaleatoria de bloquear la tarea se realizará $total\_cpu - cant\_bloqueos$ veces. 

EXPLICAR BIEN POR QUÉ ESTAMOS HACIENDO ESTO

La Figura \ref{cod-tbatch} muestra el pseudo-códgo de este algoritmo.

\begin{figure}[!htb]
\begin{codebox}
\Procname{$\proc{TaskBatch}(total\_cpu,cant\_bloqueos)$}
\li \For $i \leftarrow 0 .. (total\_cpu - cant\_bloqueos - 1)$
\li \Do 	$bloquear \leftarrow $ valor ``random'' en [0,1]
\li 		\If $bloquear == 1 \wedge $ aún hay bloqueos por hacer
\li 		\Then 	bloquear durante 1 ciclo
\li	 			decrementar $cant\_bloqueos$
\li 		\Else	usar CPU durante 1 ciclo
		\End
	\End
\li \While aún hay bloqueos por hacer
\li \Do 		bloquear durante 1 ciclo
\li 			decrementar $cant\_bloqueos$
	\End
\end{codebox}
\caption{Pseudocódigo TaskBatch}\label{cod-tbatch}
\end{figure}

\subsection{Pruebas}
\vspace*{0.6cm}

\section{Ejercicio 4}

\vspace*{0.3cm}
\section{Ejercicio 4 - Round Robin}

A continuación explicaremos nuestra implementación de un Scheduler Round Robin que permite la migración entre núcleos.

\subsection{Representación}

Hemos representado las tareas con una estructura que contiene:

\begin{itemize}
\item un entero para el pid de la tarea
\item un entero para el quantum restante de la tarea, en caso de que esté en estado {\it running}
\item un booleano que indica si la tarea está bloqueada o no
\end{itemize}

Para implementar el scheduler, hemos utilizado los siguientes atributos:

\begin{itemize}
\item un arreglo de enteros de tamaño cantidad de cores utilizados, donde cada entero representa el quantum correspondiente a dicho core
\item un arreglo de tareas del mismo tamaño, donde cada tarea representa la tarea que actualmente está corriendo en dicho core
\item una lista de tareas, correspondiente a las tareas en estado {\it ready} y {\it waiting}
\end{itemize}

\subsection{Funciones}

\paragraph{Load} Se crea una nueva tarea con el pid indicado y se agrega como último elemento de la lista de tareas {\it ready} y {\it waiting}.

\paragraph{Unblock} Se recorre la lista de tareas {\it ready} y {\it waiting} hasta encontrar a la tarea con el pid indicado, y colocarla como ``no bloqueada''.

\paragraph{Tick} Se cuenta con tres casos:

\subparagraph{TICK} Primeramente se decrementa el quantum restante de la tarea actual en el cpu indicado.  Si aún tiene quantum para correr, se devuelve su pid.  En caso de que se haya terminado su quantum, se evaluará si existe otra tarea en estado {\it ready}.  De ser así, se devolverá el pid de la primer tarea que se encuentre en dicho estado en la lista de tareas.  Si no hay otra tarea para correr, sea porque todas las demás se encuentran bloqueadas o porque no existen más tareas, se devuelve el pid de la tarea actual del cpu indicado.
\subparagraph{BLOCK} Se coloca la tarea actual del cpu indicado como ``bloqueada'', se la coloca al final de la lista de tareas y se procede a buscar la siguiente tarea a ejecutar.  Si no hay más tareas o todas están bloqueadas, se devuelve la constante IDLE_TASK.  En caso contrario, se devuelve el pid de la primer tarea en estado {\it ready} que se encuentre al recorrer la lista.
\subparagraph{EXIT} Si no hay más tareas o todas están bloqueadas, se devuelve la constante IDLE_TASK.  En caso contrario, se devuelve el pid de la primer tarea en estado {\it ready} que se encuentre al recorrer la lista.

\vspace*{0.6cm}

\section{Ejercicio 5}

\vspace*{0.3cm}
\section{Ejercicio 5 - Lote para Round Robin}

Se diseñó un lote con 3 tareas de tipo {\tt TaskCPU} de 50 ciclos y 2 de tipo {\tt TaskConsola} con 5 llamadas bloqueantes de 3 ciclos de duración cada una.  Las Figuras \ref{fig-rrq2}, \ref{fig-rrq10} y \ref{fig-rrq50} muestran los resultados de la simulación de este lote con el scheduler Round Robin implementado, con quantum de 2, 10 y 50 ciclos respectivamente.

\begin{figure}[!htb]
\begin{center}
  \includegraphics[scale=0.45]{imagenes/ej5-q2.png}
\end{center}
\caption{Simulación para SchedRR, 1 núcleo, quantum 2 y 2 ciclos de cs}\label{fig-rrq2}
\end{figure}

\begin{figure}[!htb]
\begin{center}
  \includegraphics[scale=0.45]{imagenes/ej5-q10.png}
\end{center}
\caption{Simulación para SchedRR, 1 núcleo, quantum 10 y 2 ciclos de cs}\label{fig-rrq10}
\end{figure}

\begin{figure}[!htb]
\begin{center}
  \includegraphics[scale=0.45]{imagenes/ej5-q50.png}
\end{center}
\caption{Simulación para SchedRR, 1 núcleo, quantum 50 y 2 ciclos de cs}\label{fig-rrq50}
\end{figure}

Para cada caso simulado, se calculó la latencia, el waiting-time y el tiempo total de ejecución para las 5 tareas del lote. El Cuadro \ref{tab-datos} muestra los datos obtenidos y el Cuadro \ref{tab-promedios} muestra los promedios de las 5 tareas para cada quantum.

\begin{table}[!htb]
\begin{center}
\begin{tabular}{| l | l | l | l | l |}
\hline
Task & Quantum & Latecia & Waiting-time & Total ejecución\\
\hline
0 	& 2 & 2 & 288 & 339\\
	& 10& 2 & 162 & 213\\
	& 50& 2 & 114 & 165\\
\hline
1	& 2 & 6 & 291 & 342\\
	& 10& 14& 165 & 216\\
	& 50& 54& 117 & 168\\
\hline
2	& 2 & 10 & 294 & 345\\
	& 10& 26 & 168 & 219\\
	& 50& 106& 120 & 171\\
\hline
3	& 2 & 14 & 84 & 105\\
	& 10& 38 & 201& 222\\
	& 50& 158& 177& 198\\
\hline
4	& 2 & 17 & 87 & 108\\
	& 10& 41 & 204&	225\\
	& 50& 161& 180& 201\\
\hline
\end{tabular}
\end{center}
\caption{Datos de simulación - SchedRR}\label{tab-datos}
\end{table}

%\vspace*{0.5cm}

Basándonos en los promedios calculados, el mejor caso de los simulados para latencia es el que utiliza quantum de 2 ciclos, lo cual puede explicarse dado que teniendo un quantum tan corto, todos los procesos entran en la ronda de ejecución rápidamente. Respecto a waiting-time, el mejor es el caso con quantum de 50 ciclos, y puede deberse a que al tener más ciclos por quantum el tiempo de cambio de contexto tiene menor impacto en el tiempo total de espera.  Por último, el mejor tiempo total de ejecución lo tiene también el caso con 50 ciclos de quantum, y estimamos que también tiene que ver el impacto del tiempo invertido en realizar cada cambio de contexto.

Podríamos concluir entonces que, dependiendo de las tareas que se estén ejecutando, si no es primordial que el tiempo de respuesta sea mínimo, utilizar un quantum de 50  ciclos parecería aprovechar de mejor manera los recursos del procesador. Sin embargo, si es prioritario obtener una respuesta sin importar que esto provoque  que las tareas demoren mucho tiempo en finalizarse, convendría utilizar el quantum con 2 ciclos.
\vspace*{0.6cm}

\section{Ejercicio 6}

\vspace*{0.3cm}
\section{Ejercicio 6 - Round Robin vs FCFS}

Se ejecutó una simulación con el mismo lote utilizado en la sección 5, pero esta vez con un Scheduler FCFS.  La Figura \ref{fig-ej6} muestra el gráfico de dicha simulación. Los promedios de latencia, waiting-time y tiempo total de ejecución se muestran en el Cuadro \ref{tab-promedios}.

\begin{figure}[!htb]
\begin{center}
  \includegraphics[scale=0.45]{imagenes/ej6.png}
\end{center}
\caption{Simulación para FCFS, 1 núcleo y 2 ciclos de cs}\label{fig-ej6}
\end{figure}

\begin{table}[!htb]
\begin{center}
\begin{tabular}{| l | l | l | l | l |}
\hline
Scheduler & Quantum & Latecia & Waiting-time & Total ejecución\\
\hline
RR & 2 & 9.8 & 208.8 & 247.8\\
\hline
RR & 10 & 24.2 & 180 & 219\\
\hline
RR & 50 & 96.2 & 141.6 & 180.6\\
\hline
FCFS & - & 102 & 102 & 141\\
\hline
\end{tabular}
\end{center}
\caption{Promedios - SchedRR y FCFS}\label{tab-promedios}
\end{table}

Podemos observar en la tabla de promedios del Cuadro \ref{tab-promedios}, que la latencia del FCFS es muy parecida a la del Round Robin con quantum 50, pero los valores de waiting-time y tiempo total de ejecución son significativamente menores. Considerando los resultados obtenidos en el ejercicio 5, podríamos concluir que de tener poca importancia, en cierto contexto de uso, el tiempo que una tarea tarda en empezar a ejecutar, utilizar FCFS tiene una mejor utilización de los recursos del cpu, dejándolo desocupado el menor tiempo posible (con las opciones con las cuales disponemos). Sin embargo, de ser prioritario el tiempo de respuesta de una tarea, sigue siendo más conveniente usar Round Robin con quantum 2.
\vspace*{0.6cm}

\section{Ejercicio 7}

\vspace*{0.3cm}
\section{Ejercicio 7 - Scheduler No Mistery}

\subsection{SchedMistery} 

[EXPLICAR LO QUE ENTENDIMOS]

A continuación explicaremos la implementación realizada para replicar el funcionamiento del Scheduler Mistery.

\subsection{Representación}

\subsection{Funciones}

\subsection{Pruebas}
\vspace*{0.6cm}

\section{Ejercicio 8}

\vspace*{0.3cm}
\section{Ejercicio 8 - Round Robin 2}

A continuación explicaremos nuestra implementación de un Scheduler Round Robin que no permite migración entre núcleos.

\subsection{Representación}

Para las tareas, hemos utilizado la misma estructura que en la implementación del scheduler Round Robin de la sección 4.

Para implementar el scheduler, hemos utilizado los siguientes atributos:

\begin{itemize}
\item la cantidad de cores
\item un arreglo de enteros de tamaño cantidad de cores utilizados, donde cada entero representa el quantum correspondiente a dicho core
\item un arreglo de enteros del mismo tamaño, donde cada entero representa la cantidad de tareas que están asignadas a diho core
\item un arreglo de tareas del mismo tamaño, donde cada tarea representa la tarea que actualmente está corriendo en dicho core
\item un arreglo del mismo tamaño conteniendo una lista de tareas en cada posición, donde cada lista contiene a las tareas en estado {\it ready} y {\it waiting} para cada core.
\end{itemize}

\subsection{Funciones}

\paragraph{Load} Se crea una nueva tarea con el pid indicado, se recorre el arreglo con la cantidad de tareas por cpu, y se agrega como último elemento de la lista de tareas correspondiente al cpu con menos tareas.

\paragraph{Unblock} Se recorre la lista de tareas {\it ready} y {\it waiting} de cada cpu hasta encontrar a la tarea con el pid indicado, y colocarla como ``no bloqueada''.

\paragraph{Tick} Se cuenta con tres casos:

\subparagraph{TICK} Primeramente se decrementa el quantum restante de la tarea actual en el cpu indicado.  Si aún tiene quantum para correr, se devuelve su pid.  En caso de que se haya terminado su quantum, se evaluará si existe otra tarea en estado {\it ready}.  De ser así, se devolverá el pid de la primer tarea que se encuentre en dicho estado en la lista de tareas correspondiente al cpu indicado.  Si no hay otra tarea para correr, sea porque todas las demás se encuentran bloqueadas o porque no existen más tareas, se devuelve el pid de la tarea actual del cpu indicado.
\subparagraph{BLOCK} Se coloca la tarea actual del cpu indicado como ``bloqueada'', se la coloca al final de la lista de tareas del cpu indicado y se procede a buscar la siguiente tarea a ejecutar.  Si no hay más tareas o todas están bloqueadas, se devuelve la constante IDLE_TASK.  En caso contrario, se devuelve el pid de la primer tarea en estado {\it ready} que se encuentre al recorrer la lista correspondiente.
\subparagraph{EXIT} En primer lugar, se decrementa la cantidad de tareas del cpu indicado. Si no hay más tareas o todas están bloqueadas, se devuelve la constante IDLE_TASK.  En caso contrario, se devuelve el pid de la primer tarea en estado {\it ready} que se encuentre al recorrer la lista.

\subsection{Migración vs No Migración}

La migración entre núcleos puede ser beneficiosa o no dependiendo del contexto de uso.  Por este motivo, mostraremos un escenario factible en el que la migración entre núcleos es conveniente y otro caso en el cual no lo es. Ambos casos serán planteados sin llamadas bloqueantes para una mejor comprensión.  Además, mostraremos cada escenario a través del simulador, considerando un cambio de contexto de 1 ciclo, 2 cores y un costo de migración de 4 ciclos.

En un caso en el cual se tienen algunas tareas que hacen un uso de CPU por un tiempo extenso y otras que lo utilizan durante poco tiempo, si no se utiliza migración entre núcleos podría ocurrir que las tareas extensas estén corriendo en un core y las demás en otro. La Figura \ref{fig-c2} muestra un ejemplo en el que ocurre esto, y podemos notar que el núcleo 1 rápidamente queda ocioso, mientras que el núcleo 0 sigue corriendo las tareas más pesadas. Por el contrario, la Figura \ref{fig-c1} muestra el mismo caso pero utilizando migración, y es evidente cómo se hace un mejor uso de ambos núcleos y, por lo tanto, el tiempo total de ejecución de las tareas más extensas se reduce considerablemente.  Si tomamos el tiempo total de ejecución de cada tarea en cada caso, obtenemos para la Figura \ref{fig-c2} un promedio de 54 ciclos, mientras que en la Figura \ref{fig-c1} tenemos un promedio de 32 ciclos.  Además, en el primer caso el núcleo 1 se mantuvo ocioso durante 78 ciclos, mientras que en el segundo sólo lo hizo durante 1 ciclo.

\begin{figure}[!htb]
\begin{center}
  \includegraphics[scale=0.45]{imagenes/rr2-conviene.png}
\end{center}
\caption{No migración de núcleos no beneficiosa}\label{fig-c2}
\end{figure}

\begin{figure}[!htb]
\begin{center}
  \includegraphics[scale=0.45]{imagenes/rr-conviene.png}
\end{center}
\caption{Migración de núcleos beneficiosa}\label{fig-c1}
\end{figure}

Otro caso que podría darse es que se pongan a ejecutar varias tareas cortas de manera secuencial.  Considerando que la migración entre núcleos suele ser bastante costosa, si se requieren numerosas migraciones hasta que cada tarea termina, el waiting-time de ellas se incrementaría de manera considerable, lo cual puede ser poco conveniente dado que se trata de tareas de corta duración.  Esto puede verse en la Figura \ref{fig-nc1}, mientras que, como muestra la Figura \ref{fig-nc2}, no migrar estos procesos permite que terminen en menor tiempo y, por lo tanto, disminuya su waiting-time.  Observando la Figura \ref{fig-nc1} podemos determinar un waiting-time promedio entre todas las tareas de 30,17, mientras que el mismo valor en la Figura \ref{fig-nc2} es de 14.

\begin{figure}[!htb]
\begin{center}
  \includegraphics[scale=0.45]{imagenes/rr-noconviene.png}
\end{center}
\caption{Migración de núcleos no beneficiosa}\label{fig-nc1}
\end{figure}

\begin{figure}[!htb]
\begin{center}
  \includegraphics[scale=0.45]{imagenes/rr2-noconviene.png}
\end{center}
\caption{No migración de núcleos beneficiosa}\label{fig-nc2}
\end{figure}
\vspace*{0.6cm}

\end{document}
