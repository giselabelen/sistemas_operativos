\section{Backend Multithreaded}
\subsection{Introduccion}
Debemos permitir que los jugadores se conecten en simultáneo al Scrabble, para eso utilizaremos distintos threads para atender a cada cliente que se conecte al Backend.
\subsection{Implementacion}
Ante el problema de implementar varios threads que atiendan a los clientes en simultáneo decidimos utilizar un tablero de palabras que se comparte por todos los threads. Utilizando locks sincronizaremos a los threads \\
Al tener solo un tablero compartido debemos proteger el tablero de escrituras y lecturas en simultáneo.\\
Utilizamos el modelo de sincronización clásico de escritores con acceso exclusivo y lectores con acceso compartido.\\
Nuestra implementación permite lecturas en simultáneo al tablero de palabras y elimina los tableros de letras utilizados en la implementación original.
Los principales cambios que se hicieron al código original son:
\begin{itemize}
	\item Creación de Thread al aceptar la conexión del cliente inicializado en la función atender cliente para soportar la atención de múltiples clientes.
    \item Pedidos de ReadLock y liberación del mismo en La funcion Update ya que esta esta necesita una lectura del tablero.
    \item Pedido de ReadLock y liberación del mismo en la lectura del Tablero de Palabras cuando el cliente coloca una letra.Debido a que se debe checkear que la letra no esta siendo colocada en una posición que contiene una letra en el tablero.
    \item Pedido de WriteLock y liberación del mismo cuando el cliente envía palabra al servidor.\\
    Esto es debido a que esta operación necesita colocar las letras en el tablero modificándolo.
    \item Checkeo sobre el tablero de palabras si la letra es o no válida, es decir si esta ya fue colocada o no en el tablero.
    \item Checkeo al insertar una letra o enviar una palabra si alguna de las letras del vector, que contiene la palabra, se coloco ya el tablero de palabras para evitar la escritura de palabras que ya se tornaron inválidas.
\end{itemize}


